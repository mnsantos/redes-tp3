Luego de desarrollar los experimentos a lo largo del trabajo práctico pudimos observar algunos de los problemas con los que debe lidiar el protocolo para ofrecer un servicio confiable a sus usuarios. La pérdida de paquetes y el delay en la entrega de los ACK's son algunas cosas a las que TCP y PTC deben enfrentarse todo el tiempo para lograr tal fin ya que el protocolo IP no incluye ningún monitoreo de la entrega de datagramas. Analizando el Throughput percibido (i.e., cantidad de bits por segundo) en función del tamaño de archivo, Throughput en función del delay en los ACKs (para un tamaño de archivo constante) y la cantidad de retransmisiones en función del delay en los ACKs (para un tamaño de archivo constante) observamos cómo el delay de los ACK's influye negativamente en la velocidad de transmisión. A su vez, este delay y la pérdida de paquetes generan un aumento de retransmisión de paquetes lo cual disminuye aún más la velocidad de transmision pero que asegura una entrega confiable de la información que es el principal objetivo de este protocolo.

