En este trabajo estudiaremos el transporte de paquetes a través del protocolo PTC proporcionado por la cátedra. Este protocolo está basado en TCP, pero simplificado. Sus características son que es bidireccional(full-duplex), orientado a conexión, y es confiable(usa el algoritmo de ventana deslizante). 

El objetivo del trabajo es estudiar la eficiencia y comportamiento del protocolo al transferir paquetes en una red. Como el trabajo se basará en experimentaciones sobre una red local, la cual no es muy propensa a errores y por ende no se asimila al comportamiento de una red "real", se hizo una implementación de cliente-servidor que permite simular una red "real". Esto es, simulamos los efectos de problemas comunes que existen al transferir paquetes, como son la pérdida de paquetes y la latencia.